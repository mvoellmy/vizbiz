\section{Results}
\label{sec_results}

\subsection{Bootstrapping methods}
The comparison showed a significantly higher performance of the KLT tracker approach, this being orders of magnitude faster and yielding many more feature correspondences also over large distances. \cref{tab_boot_idx} depicts the index tuples retrieved through the second approach with \code{min\_num\_inlier\_kp} = $600$ being required.
\begin{center}
	\begin{tabular}{ c | l | c  c  c }
		approach &				 &				Kitti &		Malaga &	Parking\\
  		\hline
 \multirow{2}{*}{\textit{KLT + Bearing angle}} &	first idx &	1 & 		1 & 		1\\
  		 & 						second idx &	4 & 		6 & 		5\\
  		& 						\# keypoints & 				600 & 		600 & 		600\\
  		&						angle [deg] & 				1.37 &			4.59 &			4.07\\
  		\hline
	\end{tabular}
	\captionof{table}{Bootstrapping pair indices for different datasets with minimum keypoint inliers}
	\label{tab_boot_idx}
\end{center}

When relaxing the minimum number of inlier keypoints allowed to reach the desired baseline/depth ratio of $10$ \% for the first and a bearing angle of $10$ degrees for the second bootstrapping approach results are shown in \cref{tab_relaxed_boot_idx}.
\begin{center}
	\begin{tabular}{ c | l | c  c  c }
		approach &				 &				Kitti &		Malaga &	Parking\\
  		\hline
 \multirow{4}{*}{\textit{SSD Harris desc. + baseline/depth ratio}} 
 		&						first idx &			1 & 		1 & 		1\\
  		& 						second idx &		6 & 		10 & 		7\\
  		& 						\# keypoints & 		34 & 		39 & 		54\\
  		&						ratio [\%] & 		10.1 &		10.9 &		11.5\\
  		\hline
  \multirow{4}{*}{\textit{KLT + bearing angle}} 
  		&						first idx &			- & 		1 & 		1\\
  		& 						second idx &		- & 		16 & 		17\\
  		& 						\# keypoints & 		- & 		99 & 		74\\
  		&						angle [deg] & 		- &			10.4 &		10.55\\
  		\hline
	\end{tabular}
	\captionof{table}{Bootstrapping pair indices for different datasets and no keypoint number constraint}
	\label{tab_relaxed_boot_idx}
\end{center}

We observed that the first approach yields far too few keypoints, due to the matchDescriptor() method. Furthermore, the pure lateral camera motion of dataset Parking suits well this baseline/depth bootstrapping approach, which confirms our intuition given the 'triangulability condition' described in \cref{sec_boot}.\\

Deploying the second approach on the Kitti dataset evidently fails due to the straight camera motion, inhibiting angles ($[0.5, 3.7]$ deg) to reach the desired 10 degrees. As soon as there is a distinctive rotational movement involved, e.g. in Malaga and Parking, a useful bootstrapping is performed.


\subsection{Overall performance}
(keywords: Real time ness, comparison to groundtruth, compare different datasets
Impact of features) Here we describe it such as it runs best in our opinion.

\subsubsection{Key parameters}

\begin{table}[!h]
	\centering
	\begin{tabular}{|l|c|c|c|c|c|}
	\hline
	\multicolumn{1}{|c|}{\textbf{Parameter / Dataset}} & \textbf{Kitti} & \textbf{Malaga} & \textbf{Parking} & \textbf{Poly up} & \textbf{Poly down} \\ \hline
	\# landmarks reinit                                & 25             &                 & 80               &                  &                    \\ \hline
	\# min bearing angle {[}deg{]}                     & 1.4            &                 & 3.7              &                  &                    \\ \hline
	bearing angle threshold {[}\# landmarks{]}         & 180            &                 & 230              &                  &                    \\ \hline
	RANSAC tolerance {[}pixel{]}                       & 2              &                 & 2                &                  &                    \\ \hline
	\# candidate keypoints in tracker                  & 1300           &                 & 1300             &                  &                    \\ \hline
	\end{tabular}
	\caption{Key parameters chosen for every dataset}
	\label{params_table}
\end{table}

This list will be removed once table is completed
\begin{compactitem}
	\item Kitti @Pascal
	\item Malaga @Milan
	\item Parking @Pascal
	\item Poly-up @ Fabio
	\item Poly-down @Fabio
\end{compactitem}
	
	
\subsubsection{Runtime characteristics}
(e.g. average nr inlier landmarks, and new landmarks, candidate kp)

\subsubsection{plot of groundtruth}

\subsubsection{3d landmarks if possible}

\subsubsection{speed \& realtimeness}
slowest functions - why?


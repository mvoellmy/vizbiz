\subsection{Initialization}
\label{sec_init}
Todo: @Miro
%short intro

\subsubsection{Bootstraping}
\label{sec_boot}
In order to pick reasonable image pairs for initialization an automatic procedure was implemented.\\
Since a good initialization relies heavily on the number of inlier keypoints a hard constrain on their minimum number of \code{min\_num\_inlier\_kp} = $600$ was set.\\

Two approaches were investigated:
\begin{compactitem}
	\item \textit{SSD matching + Baseline/Depth}:\\
	correspondence search via SSD on Harris features \& 8-point-Ransac for outlier rejection \& baseline/depth-ratio (as proposed in the lecture)\\
	\begin{equation}
		\frac{C1\_baseline}{C1\_av\_depth} \geqslant min\_b2dratio = 0.1
	\end{equation}\\
	
	\item \textit{KLT + bearing angle}:\\
	correspondence search and inlier selection via KLT tracking on query Harris corners \& minimum average bearing angle\\
	\begin{equation}
		av\_bearing\_angle\_deg \geqslant min\_av\_angle\_deg = 10 deg
	\end{equation}
\end{compactitem}

As elaborated in \cref{sec_results} the second approach outperformed the other and is selected as auto-bootstrapping method.


\subsubsection{? findCorrespondences()}
Todo: @Miro
%write one sentence about using either harris matching(SSD?) or klt tracking - any %different to findCorrespondences_cont()?

\subsubsection{eightPointRansac}
Todo: @Miro

\subsubsection{linearTriangulation}
maybe only 1 sentences - ``like in the exercise''

\subsubsection{bundleAdjust}
@Miro
is here anything different then in the continuous bundle adjust?

\subsubsection{applySphericalFilter}
\label{sub_sec_sphFilter}
Discards all landmarks which are not within a half-sphere in front of the camera. This removes on one hand landmarks triangulated behind the camera (negative z-components) and also landmarks further away than the filter cutoff radius. With this filtering only landmarks in the neighborhood are kept, which are more reliably triangulated (smaller uncertainty cone). Note, that this absolute thresholding relies on an accurate scale estimation.

\begin{figure}[ht]
	\centering
	\includegraphics[width=0.8\textwidth]{init/init_chart}
	\caption{Init Flow chart}
	\label{img_flow_init}
\end{figure}